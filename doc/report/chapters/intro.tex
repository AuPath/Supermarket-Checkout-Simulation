\chapter{Introduzione}

In questo progetto è stato costrutio un modello basato su aganti con lo scopo di simulare il comportamento dei clienti all'interno di un supermercato durante la fase di scelta della coda relativa a diversi tipi di casse.

Un grande supermercato è composto molte casse, ognuna di queste ha un comportamento diverso. \todo{Inserire il riferimento al capitolo} Come vederemo in dettaglio nel capitolo successivo, esistono diversi tipi di casse:

il tipo standard, in cui ogni cassa ha la relativa coda, tante casse con un'unica coda condivisa, casse in cui non è presente un cassiere ma è il cliente stesso a scannerizzare i prodotti a fine spesa e utimamente sono comaprsi nei suepermercati delle casse in cui il cliente scanerizza i prodotti durante la spesa così da arrivare effettuare solamente il pagamento una volta arrivato in cassa.

Dal momento che in un grande supermercato sono presenti più clienti che casse, è fondamentale l'utilizzo di code per gestire la grande quantità di clienti.  Un supermercato può essere considerato a tutti gli effetti un sistema complesso in qunato si hanno diversi aspetti che considerati contemporaneamnte fanno emergerge un comportamento complessivo difficilmente prevedibile, tra questi aspetti si hanno:
\begin{itemize}
	\item Flusso di clienti in ingresso variabile (influendo sulle persone in coda)
	\item Numero di prodotti che un cliente acquista durante la spesa (influendo sul tempo passato in cassa e quindi scatenando eventualmente il cambio della coda da parte di altri clienti)
	\item Il numero di casse aperte contemporaneamente nel supermercato
	\item Strategia di scelta della coda dei clienti (un cliente può avere diversi criteri per la scelta della coda)
	\item Stategia di cambio della coda se ne esiste una più conveniente
\end{itemize}


Lo scopo di questo lavoro è costruire un modello basato su agenti per simulare il comprtamento del supermercato. Dopo una prima fase di validazione, sfruttando i pochi dati a disposizione, vengono considerare diversi casi "what-if" con lo scopo di sperimentare diverse configuarzioni di casse per una gestione ottimale del flusso dei clienti. 
