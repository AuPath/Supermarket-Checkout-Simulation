\chapter{Introduzione}

In questo progetto è stato costruito un modello basato su agenti con lo scopo di simulare il comportamento dei clienti all'interno di un supermercato durante la fase di scelta della coda relativa a diversi tipi di casse.

Un grande supermercato è composto da molte casse diverse, ognuna di queste ha un comportamento diverso. Come vederemo in dettaglio nel capitolo \ref{chapter:model}, esistono diversi tipi di casse: la cassa standard, in cui si trova un cassiere che passa i prodotti provenienti dal nastro, la cassa self-service, in cui è il cliente stesso a scannerizzare i prodotti e la cassa self-scan, comparsa negli ultimi anni, in cui il cliente deve soltanto pagare, in quanto ha già effettuato la scannerizzazione dei prodotti man mano che li ha raccolti nel carrello. La configurazione delle casse nel supermercato può prevedere che ogni cassa abbia la sua coda dedicata, oppure che tante casse condividano la stessa coda.

Dal momento che in un grande supermercato sono presenti più clienti che casse, è fondamentale l'utilizzo di code per gestire la grande quantità di clienti.  Un supermercato può essere considerato a tutti gli effetti un sistema complesso in quanto si verificano diversi aspetti che considerati contemporaneamente fanno emergere un comportamento complessivo difficilmente prevedibile, tra questi:
\begin{itemize}
	\item Flusso di clienti in ingresso variabile (che influisce sulle persone in coda)
	\item Numero di prodotti che un cliente acquista durante la spesa (che influisce sul tempo passato in cassa, scatenando eventualmente il cambio della coda da parte di altri clienti)
	\item Numero di casse aperte contemporaneamente nel supermercato
	\item Strategia di scelta della coda dei clienti (un cliente può avere diversi criteri per la scelta della coda)
	\item Strategia di cambio della coda dei clienti
\end{itemize}


Lo scopo di questo lavoro è costruire un modello basato su agenti per simulare il comportamento del supermercato. Dopo una prima fase di validazione, sfruttando i pochi dati a disposizione, vengono considerati diversi casi "what-if" con lo scopo di sperimentare diverse configurazioni di casse e strategie di scelta della coda per una gestione ottimale del flusso dei clienti. 

Il codice sorgente per il modello e i notebbok utilizzati con lo scopo di analisi dei risultati si possono trovare sul repository al seguente link:
\url{https://github.com/AuPath/Supermarket-Checkout-Simulation}

Questa relazione è organizzata come segue: nel capitolo \ref{chapter:sota} vengono descritte brevemente le fonti da cui prende ispirazione il progetto e giustificate le scelte riguardo i tipi di casse e le strategie usate dai clienti. Nel capitolo \ref{chapter:model} sono descritti il sistema supermercato e gli agenti, con rimandi alla teoria e una descrizione in dettaglio dell'implementazione dei componenti; in questo capitolo vengono anche descritti i parametri usati nel modello che possono essere cambiati a piacere. Nel capitolo \ref{implementation:intro} vengono descritte le strategie di scelta della coda e di cambio di coda (jockeying); le strategie vengono introdotte molto dettagliatamente perchè rivestono un ruolo importante nel progetto, infatti l'obiettivo finale è capire quale di esse porta a un guadagno maggiore in termini di tempi di attesa in cassa. Nel capitolo \ref{chapter:simulation} si analizzano le simulazioni del modello, in particolare la prima simulazione servirà da confronto con l'articolo da cui prende principalmente spunto questo progetto; le altre simulazioni serviranno quindi a indagare sugli elementi che abbiamo voluto introdurre per espandere e rendere più realistico il nostro supermercato.
