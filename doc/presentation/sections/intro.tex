\section{Introduzione}
\begin{frame}{Introduzione}
	
	Il nostro progetto è un modello basato su agenti che \textbf{simula} il comportamento dei clienti in un supermercato durante le fasi di scelta della coda relativa a diversi tipi di casse.
	
	Prenderemo in considerazione 3 tipi di casse diverse (standard, self-service e self-scan) e 2 tipi di scelte fatte dai clienti (scelta della coda, jockeying).
	
	L'utilizzo di code è fondamentale per gestire le grandi quantità di clienti.
	
	\begin{block}{Obiettivo}
		Sperimentare diverse \textbf{configurazioni di casse} e \textbf{strategie di scelta della coda} per gestire in modo ottimale il flusso di clienti e ridurre al minimo il tempo d'attesa passato in coda.
	\end{block}
\end{frame}

\begin{frame}
	Un supermercato è un \textbf{sistema complesso} in cui agiscono diverse entità, come clienti e casse. 
	
	Si verificano aspetti emergenti difficilmente prevedibili dovuti a diversi aspetti:
	\begin{itemize}
		\item Flusso di clienti in ingresso variabile
		\item Numero di prodotti che un cliente acquista
		\item Numero di casse aperte contemporaneamente
		\item Strategia di scelta della coda dei clienti
		\item Strategia di cambio della coda dei clienti
	\end{itemize}
\end{frame}