\section{Implementazione dei comportamenti e delle strategie}


%----------------------------------------------------------------------------------------



\begin{frame}{Implementazione dei comportamenti e delle strategie}
	\begin{itemize}
		\item Il cliente una volta presi tutti gli articoli si reca alla coda.
		\item La cassa self-scan deve essere scelta prima di fare la spesa e non è possibile cambiare.
		\item Il cliente vuole minimizzare il tempo speso all'interno del supermercato tramite:
		\begin{itemize}
			\item \textbf{Scelta iniziale della coda}: il cliente sceglie la coda ottima rispetto ad una determinata strategia
			\item \textbf{Fase di jockeying}: una volta in coda il cliente può scegliere di cambiarla se ne esiste una migliore
		\end{itemize}
	\end{itemize}
\end{frame}


%----------------------------------------------------------------------------------------
\begin{frame}{Scelta della coda}
  \centering

  \begin{itemize}
  \item Terminata la fase di spesa un cliente decide tra le casse
    aperte dove accodarsi individuando la coda ottimale per una
    determinata metrica.

  \item A questo comportamento fanno eccezione i clienti che hanno
    inizialmente optato per la modalitá di spesa self scan per i quali
    é obbligatorio recarsi alle casse di tipo self scan.

  \item Un cliente puó accodarsi ad una cassa self-service solo se ha
    un numero di prodotti inferiore al limite imposto.
  \end{itemize}
\end{frame}


% ----------------------------------------------------------------------------------------
\begin{frame}{Strategie di scelta della coda 1-2}
  La scelta della coda puó avvenire in base a 4 strategie:
  \begin{enumerate}
  \item Minor numero di elementi

    \begin{equation}
      \argmin_q \sum\limits_{i=1}^N \text{estimate-basket-size}(c_i) 
    \end{equation}

    Per rendere piú realistico il calcolo é possibile "sbagliare"
    il conto di articoli per cliente tramite il parametro
    \textit{standard deviation coefficient}.
    
  \item Minor numero di persone
   \end{enumerate}
   
   \begin{equation}
     \argmin_q |q|
   \end{equation}

   Dove \textit{q} é una coda e c\textsubscript{i} é l'i-esimo
   cliente.   
 \end{frame}

% ----------------------------------------------------------------------------------------
\begin{frame}{Strategie di scelta della coda 3-4}
  \begin{enumerate}
    \setcounter{enumi}{2}

  \item Minor tempo di attesa rispetto al tempo di servizio medio
    \begin{equation}
      \argmin_q |q| * \frac{1}{M}\sum\limits_{j=1}^M \left( \text{total-service-time}(q_j) \right)
    \end{equation}

    Dove \textit{M} é il numero di code nel supermercato
    
\item Minor tempo di attesa rispetto alla \textit{power regression}
  \begin{equation}
    \argmin_q |q| * \text{total-service-time}(q)
  \end{equation}
\end{enumerate}
\end{frame}

% ----------------------------------------------------------------------------------------
\begin{frame}{Formule cassa standard}
	\begin{itemize}
		\item Transaction time
		\begin{equation}
			\text{transaction-time}_i = e^{a log(\text{estimate-basket-size}(c_i)) + b}
		\end{equation}
		\item Break Time
		\begin{equation}
			\text{break-time}_i = \frac{\beta^{\alpha} \text{estimate-basket-size}(c_i)^{\alpha - 1} e^{- \beta \text{estimate-basket-size}(c_i)}}{\Gamma (\alpha)}
		\end{equation}  
		
		\item Dove $\Gamma$ è la funzione \textbf{gamma}
		\begin{equation}
			\Gamma (z) = \int_{0}^{\infty} x^{z-1} e^{-x} dx \;\; \forall z \in \mathbb{C}
		\end{equation}
		
		\item Total service time
		\begin{equation}
			\text{total-service-time}(q_j) = \sum\limits_{i=1}^N \left( \text{transaction-time}_i + \text{break-time}_i \right)
		\end{equation}
		
		\begin{equation}
			a = 0.6984, \;\; b = 2.1219, \;\; \alpha = 3.074209, \;\; \beta = \frac{1}{4.830613}
		\end{equation}
		
	\end{itemize}  
\end{frame}

% ----------------------------------------------------------------------------------------
\begin{frame}{Formule cassa self-service}
	\begin{itemize}
		\item Transaction time
		\begin{equation}
			\text{transaction-time}_i = e^{a log(\text{estimate-basket-size}(c_i)) + b}
		\end{equation}
		
		\item Break Time
		\begin{equation}
			\text{break-time}_i = e^{c log(\text{estimate-basket-size}(c_i)) + d}
		\end{equation}
		\item Total service time
		\begin{equation}
			\text{total-service-time}(q_j) = \sum\limits_{i=1}^N \left( \text{transaction-time}_i + \text{break-time}_i \right)
		\end{equation}
		
		\begin{equation}
			a = 0.6725, \;\; b = 3.1223, \;\; c = 0.2251, \;\; d = 3.5167
		\end{equation}
		
	\end{itemize}  
\end{frame}

% ----------------------------------------------------------------------------------------


\begin{frame}{Jockeying}
	\begin{itemize}
		\item Un cliente fa \textbf{jockeying} se calcola che nelle code adiacenti a quella in cui è in attesa c'è un tempo di attesa minore, e quindi si sposta.
		\item Il \textit{parametro di adiacenza} determina il numero di code adiacenti che il cliente prende in considerazione per il suo calcolo.
		\item Il cliente calcola un \textit{guadagno} di tempo nel cambiare coda, se questo supera un certo threshold, allora fa jockey, altrimenti no perchè per lui "non ne vale la pena".
		\item Anche se esistono code migliori di altre, può non avvenire il jockey: viene estratto un parametro che rende il jockey aleatorio, in quanto non tutte le persone lo fanno.
	\end{itemize}
\end{frame}

\begin{frame}{Jockeying - strategie}
	\begin{itemize}
		\item Sono 2 le strategie per fare jockeying: 
		\begin{enumerate}
			\item \textbf{Minimo numero di elementi}: è scelta la coda con il minor numero di elementi nei carrelli di tutti i clienti. Il guadagno è:
			\[g = \# \text{elementi nei carrelli nella coda pivot} - \min\limits_{q \in Q_{adj}} \# \text{elementi nei carrelli}\]
			\item \textbf{Minimo numero di persone}: è scelta la coda con il minor numero di persone accodate. Il guadagno è:
			\[g = i - \min\limits_{q \in Q_{adj}} |q|\]
			dove $i$ è la posizione del cliente nella coda pivot
		\end{enumerate}
	\end{itemize}
\end{frame}