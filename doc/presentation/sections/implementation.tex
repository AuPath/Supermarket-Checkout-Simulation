\section{Implementazione dei comportamenti e delle strategie}


%----------------------------------------------------------------------------------------



\begin{frame}{Implementazione dei comportamenti e delle strategie}
	\begin{itemize}
		\item Il cliente una volta presi tutti gli articoli si reca alla coda.
		\item La cassa self-scan deve essere scelta prima di fare la spesa e non è possibile cambiare.
		\item Il cliente vuole minimizzare il tempo speso all'interno del supermercato tramite:
		\begin{itemize}
			\item \textbf{Scelta iniziale della coda}: il cliente sceglie la coda ottima rispetto ad una determinata strategia
			\item \textbf{Fase di jockeying}: una volta in coda il cliente può scegliere di cambiarla se ne esiste una migliore
		\end{itemize}
	\end{itemize}
\end{frame}


%----------------------------------------------------------------------------------------



\begin{frame}{Scelta della coda}
	\centering
	definizione di strategia (con formula argmin) - 4 strategie - formule delle strategie
	\#TODO: Marco

\end{frame}


%----------------------------------------------------------------------------------------



\begin{frame}{Jockeying}
	\begin{itemize}
		\item Un cliente fa \textbf{jockeying} se calcola che nelle code adiacenti a quella in cui è in attesa c'è un tempo di attesa minore, e quindi si sposta
		\item Il parametro di adiacenza determina il numero di code adiacenti che il cliente prende in considerazione per il suo calcolo
		\item Il cliente calcola un \textit{guadagno} di tempo nel cambiare coda, se questo supera un certo threshold, allora fa jockey, altrimenti no perchè per lui "non ne vale la pena"
		\item Anche se esistono code migliori di altre, può non avvenire il jockey: viene estratto un parametro che rende il jockey aleatorio, in quanto non tutte le persone lo fanno
	\end{itemize}
\end{frame}

\begin{frame}{Jockeying - strategie}
	\begin{itemize}
		\item Sono 2 le strategie per fare jockeying: 
		\begin{enumerate}
			\item \textbf{Minimo numero di elementi}: è scelta la coda con il minor numero di elementi nei carrelli di tutti i clienti. Il guadagno è:
			\[g = \# \text{elementi nei carrelli nella coda pivot} - \min\limits_{q \in Q_{adj}} \# \text{elementi nei carrelli}\]
			\item \textbf{Minimo numero di persone}: è scelta la coda con il minor numero di persone accodate. Il guadagno è:
			\[g = i - \min\limits_{q \in Q_{adj}} |q|\]
			dove $i$ è la posizione del cliente nella coda pivot
		\end{enumerate}
	\end{itemize}
\end{frame}