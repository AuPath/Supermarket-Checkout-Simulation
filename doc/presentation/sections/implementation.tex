\section{Implementazione dei comportamenti e delle strategie}


%----------------------------------------------------------------------------------------



\begin{frame}{Implementazione dei comportamenti e delle strategie}
	\begin{itemize}
		\item Il cliente una volta presi tutti gli articoli si reca alla coda.
		\item La cassa self-scan deve essere scelta prima di fare la spesa e non è possibile cambiare.
		\item Il cliente vuole minimizzare il tempo speso all'interno del supermercato tramite:
		\begin{itemize}
			\item \textbf{Scelta iniziale della coda}: Il cliente sceglie la coda ottima rispetto ad una determinata strategia.
			\item \textbf{Fase di jockeying}: Una volta in coda il cliente può scegliere di cambiarla se ne esiste una migliore.
		\end{itemize}
	\end{itemize}
\end{frame}


%----------------------------------------------------------------------------------------



\begin{frame}{Scelta della coda}
	\centering
	definizione di strategia (con formula argmin) - 4 strategie - formule delle strategie
	\#TODO: Marco

\end{frame}


%----------------------------------------------------------------------------------------



\begin{frame}{Jockeying}
	\centering
	adiacenza - threshold - probabilità di non fare jockeying - 2 strategie e formule per il calcolo del guadagno
	\#TODO: Lucrezia
\end{frame}