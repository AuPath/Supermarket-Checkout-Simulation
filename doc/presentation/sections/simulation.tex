\section{Simulazione}
\begin{frame}{Simulazione}
	Introduciamo le definizioni di densità e flusso di clienti ad ogni step al fine di avere una misura della gestione dell'affluenza di clienti nel negozio per le simulazioni.
	
	La \textbf{densità} di clienti per step corrisponde al numero di clienti medio per ogni cassa; la densità allo step $i$ è:
	\[\text{density}_i = \frac{\# \text{ customers in the supermarket}}{\# \text{ cashdesks}}\]
	
	Il \textbf{flusso in entrata} di clienti per step corrisponde al numero di clienti che entrano ad ogni step per ogni cassa in media; il flusso allo step $i$ è:
	\[\text{flow}_i = \frac{\# \text{ customers entering in the supermarket}}{\# \text{ cashdesks}}\]
\end{frame}

\begin{frame}{Validazione}
	\centering
	Stessa simulazione dei polacchi con le 4 strategie
\end{frame}

\begin{frame}{Simulazione con jockey}
	\centering
	Uguale a sopra aggiungendo il jockey (4 strategie + 2 strategie di jockey)
\end{frame}

\begin{frame}{Simulazione con coda condivisa}
	\centering
	Uguale a sopra ma con coda condivisa (quindi senza scelta della coda e senza jockey)
\end{frame}

\begin{frame}{Simulazione con casse self-scan}
	\centering
	Non importa che sia uguale a sopra perchè tanto è come se fosse una simulazione a parte, si può fare anche con 0 standard e 0 self-service, giusto per non avere mille robe in mezzo
\end{frame}

\begin{frame}{Simulazione non deterministica}
	\centering
	Uguale con tutto (quindi a sx abbiamo i self-scan e a dx le casse normali con coda condivisa o meno, 4 strategie di scelta della coda, 2 strategie di jockey) però accendiamo i parametri probabilistici 
\end{frame}